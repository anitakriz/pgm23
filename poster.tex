%%%%%%%%%%%%%%%%%%%%%%%%%%%%%%%%%%%%%%%%%
% NIWeek 2014 Poster by T. Reveyrand
% www.microwave.fr
% http://www.microwave.fr/LaTeX.html
% ---------------------------------------
% 
% Original template created by:
% Brian Amberg (baposter@brian-amberg.de)
%
% This template has been downloaded from:
% http://www.LaTeXTemplates.com
%
% License:
% CC BY-NC-SA 3.0 (http://creativecommons.org/licenses/by-nc-sa/3.0/)
%

%%%%%%%%%%%%%%%%%%%%%%%%%%%%%%%%%%%%%%%%%

%----------------------------------------------------------------------------------------
%	PACKAGES AND OTHER DOCUMENT CONFIGURATIONS
%----------------------------------------------------------------------------------------

\documentclass[a0paper,portrait]{baposter}

\usepackage[font=small,labelfont=bf]{caption} % Required for specifying captions to tables and figures
\usepackage{booktabs} % Horizontal rules in tables
\usepackage{relsize} % Used for making text smaller in some places
\usepackage{wrapfig}
\usepackage{amsmath,amsfonts,amssymb,amsthm} % Math packages
\usepackage{eqparbox}
\usepackage{ulem}
\usepackage{textcomp}
\usepackage{enumitem}
\graphicspath{{figures/}} % Directory in which figures are stored

 \definecolor{bordercol}{RGB}{40,40,40} % Border color of content boxes
 \definecolor{headercol1}{RGB}{186,215,230} % Background color for the header in the content boxes (left side)
 \definecolor{headercol2}{RGB}{75,0,130} % Background color for the header in the content boxes (right side)
 \definecolor{headerfontcol}{RGB}{0,0,0} % Text color for the header text in the content boxes
 \definecolor{boxcolor}{RGB}{255,255,255} % Background color for the content in the content boxes


\begin{document}

\background{ % Set the background to an image (background.pdf)
\begin{tikzpicture}[remember picture,overlay]
\draw (current page.north west)+(-2em,2em) node[anchor=north west]
{\includegraphics[height=1.1\textheight]{background}};
\end{tikzpicture}
}

\begin{poster}{
grid=false,
borderColor=bordercol, % Border color of content boxes
headerColorOne=headercol1, % Background color for the header in the content boxes (left side)
headerColorTwo=headercol2, % Background color for the header in the content boxes (right side)
headerFontColor=headerfontcol, % Text color for the header text in the content boxes
boxColorOne=boxcolor, % Background color for the content in the content boxes
headershape=roundedright, % Specify the rounded corner in the content box headers
headerfont=\Large\sf\bf, % Font modifiers for the text in the content box headers
textborder=rectangle,
background=user,
headerborder=open, % Change to closed for a line under the content box headers
boxshade=plain
}
{\includegraphics[scale=0.05]{figures/udem.png}}
%
%----------------------------------------------------------------------------------------
%	TITLE AND AUTHOR NAME
%----------------------------------------------------------------------------------------
%
{ \bf  \Huge {Rethinking Belief Propagation: \\Can Graph Neural Networks Take the Lead?} } % Poster title
{\vspace{0.25em} \smaller Anita Kriz$^1$, Clemence Granande$^1$, Anthony Gosselin$^1$, Jeremy Qin$^1$, Alireza Dizaji$^1$   \\  % Author names

\smaller $^1$\it {Mila} \\ } 
{\includegraphics[scale=0.3]{figures/Mila.png}} % University/lab logo
%----------------------------------------------------------------------------------------
%	INTRODUCTION
%----------------------------------------------------------------------------------------

\headerbox{Introduction}{name=introduction,column=0,row=0, span=3}{
\centering 
\textbf{Problem: Performing inference tasks on graphs is HARD} 

\begin{minipage}[t]{0.20\textwidth}
    \centering\textbf{} \\
    \includegraphics[width=1.2\textwidth]{figures/bp.png} % Replace 'example-image-a' with your figure file name and extension
\end{minipage}%
\hfill
\begin{minipage}[t]{0.25\textwidth}
    \textbf{\textit{It is possible on trees...}}
    \vspace{2pt} \\
    % \textbf{To compute marginals for inference:} \\
    \textbf{X} \uline{{Naïve Marginalization:}} in $O(k^{|N|})$ 
    \begin{align}
        P(x_i) &= \sum_{x_1 \in X_1}\ldots \sum_{x_{N} \in X_{N}} P(x_1, x_2, \ldots, x_{N}) \nonumber
    \end{align}
    \checkmark \uline{Belief Propagation (BP)} in  $O(|N|)$:
    \begin{itemize}[itemsep=.5pt]
        \item \textbf{Initiate} message passing at leaves
        \item \textbf{Propagate and Store} messages up to the root and back
    \end{itemize} 
\end{minipage}
\hfill
\begin{minipage}[t]{0.18\textwidth}
    \centering\textbf{} \\
    \includegraphics[width= 1.2\textwidth]{figures/bp_cycle.png} % Replace 'example-image-a' with your figure file name and extension
\end{minipage}%
\hfill
\begin{minipage}[t]{0.3\textwidth}
\textbf{\textit{... but not on graphs in general}}
\vspace{1pt} \\
 Can \textbf{approximate}: run BP for $t$ iterations, BUT:
\begin{itemize}
    \item May not converge, even as $t \to \infty$ 
    \item May not have a closed-form solution 
    \item May not be accurate due to the complex dependencies in graph 
\end{itemize} 
\end{minipage} \\
\centering \textbf{Research Question: How can we find better approximators for inference tasks in Probabilistic Graphical Models (PGMs)?}
}


%----------------------------------------------------------------------------------------
%	BACKGROUND
%----------------------------------------------------------------------------------------
\headerbox{Background}{name=background,column=0, below=introduction}{

\subsection*{Belief Propagation}
Can be generalized to graphs with loops, i.e. \textit{loopy BP}:
\begin{equation}
        \mu_{i \to \alpha}^{(t)}(X_i) = \prod_{\beta \in N_i \setminus \alpha} \mu_{\beta \to i}^{(t-1)}(X_i)
        \end{equation}
\begin{equation}
\mu_{\alpha \to i}^{(t)}(X_i) = \sum_{\substack{X_{\alpha} \setminus X_i}} \psi_{\alpha}(X_{\alpha}) \prod_{j \in N_\alpha \setminus i} \mu_{j \to \alpha}^{(t-1)}(X_j)
\end{equation}
where $N_i$ are the neighbors of variable node $X_i$ and $N_\alpha$ are the neighbors of factor nodes $\alpha$ 
\subsection*{GNNs}
Use a message passing method $\bm{m}^{(t+1)}_{i\rightarrow}$ to obtain the hidden vector states $\bm{h}_i^{(t+1)}$ for a GNN node $v_i$ at time $t+1$:
\begin{equation}
    \bm{m}_{i}^{(t)} = \sum_{j \in N(i)} \bm{m}^{t+1}_{j \to i} =  
    \sum_{j \in N(i)} \mathcal{M}\left(\bm{h}^t_j, \bm{h}^t_i, e_{ji}\right)
\end{equation}
\begin{align*}
    \bm{h}_i^{(t+1)} = \mathcal{U}(\bm{h}_i^{(t)}, \bm{m}_{i}^{(t+1)})
\end{align*}
% \begin{itemize}
%     \item $\bm{h}_i^{(t+1)}$: hidden vector states for a GNN node $v_i$ at time $t+1$
%  \item $\bm{m}^{(t+1)}_{i\rightarrow}$: message from GNN node $v_i$ to connect GNN node $v_j$ sent at time $t+1$
%  \item $\mathcal{M}$: MLP with ReLu
%  \item $e_{i, j}$: edge labels/properties between the nodes
%  \item $N(i)$: indices of neighboring nodes for node $v_i$
%  \item $\mathcal{U}$: update function (GRU or LSTM)
%  \item $\hat{y}$: output marginals
% \end{itemize}
Obtain the marginal distribution at $T$:
\begin{align*}
    \bm{\hat{y}} = \sigma (\bm{h}^{(T)})
\end{align*}
Trained with $L(\bm{\hat{y}}, \bm{y}) = -\bm{y}_i(x_i)\log(\bm{\hat{y}}_i(x_i))$

\textbf{$\Rightarrow$ Update Functions}

\begin{center}
\minipage{0.45\textwidth}
\begin{center}
\textbf{LSTM}
\includegraphics[width=\linewidth]{./poster_imgs/LSTM.png}
% \begin{align*}
%     i_t = 
%     \bm{h}_i^{(t+1)} = o_t * tanh(c_t) 
% \end{align*}
\end{center}
\endminipage\hfill
\minipage{0.45\textwidth}
\begin{center}
\textbf{GRU}
\includegraphics[width=\linewidth]{./poster_imgs/GRU.png}
\end{center}
\endminipage\hfill
\end{center}


\textbf{$\Rightarrow$ Soft Attention Mechanism}

\minipage{0.32\textwidth}
\begin{center}
\includegraphics[width=\linewidth]{./poster_imgs/soft_attention.png}
\end{center}
\endminipage\hfill
\minipage{0.32\textwidth}\begin{center}
\begin{align*}
    c'_t &= \text{LeakyRelu}(e_{i,j}*A^\intercal + b) \\
    c_t &= \text{softmax}(c'_t)\\
    e_{i,j} &= c_t * e'_{i,j} \\
\end{align*}
\end{center}\endminipage\hfill


% \begin{figure}[ht]
%     \centering
%     \begin{subfigure}{0.3\linewidth}
%         \centering
%         \includegraphics[width=\linewidth]{./poster_imgs/Attention.png}
%         \caption{Caption for Image 1}
%     \end{subfigure}
%     \hspace{1cm}
%     \begin{subfigure}{0.3\linewidth}
%         \centering
%         \includegraphics[width=\linewidth]{./poster_imgs/Attention.png}
%         \caption{Caption for Image 2}
%     \end{subfigure}
%     \hspace{1cm}
%     \begin{subfigure}{0.3\linewidth}
%         \centering
%         \includegraphics[width=\linewidth]{./poster_imgs/Attention.png}
%         \caption{Caption for Image 3}
%     \end{subfigure}
%     \caption{Overall Caption for the Figure}
%     \label{fig:your_figure_label}
% \end{figure}
}


%----------------------------------------------------------------------------------------
%	MAPPING
%----------------------------------------------------------------------------------------
\headerbox{PGM to GNN Mapping}{name=mapping,span=2,column=1,row=1, below=introduction}{ % To reduce this block to 1 column width, remove 'span=2'
\begin{minipage}[t]{0.25\textwidth}
    \centering\textbf{} \\
    \includegraphics[width = .8\textwidth]{figures/gnn_mapping.png}
\end{minipage}
\hfill
\begin{minipage}[t]{0.36\textwidth}
    \textsc{Message Node Mapping}\\
    Message updates:
    \begin{equation}
    m^{(t+1)}_{i \to j} = \mathcal{M}\left(\sum_{k \in N_{i} \setminus {j}} h^t_{k \to i}, e_{ij}\right)
    \end{equation}
    Hidden State Updates:
    \begin{equation}
    h_{i \to j}^{(t+1)} = \mathcal{U}(h_{i \to j}^{(t)}, \bm{m}_{i \to j}^{(t + 1)})
    \end{equation}
    Node Marginals:
    \begin{equation}
    \hat{p}_i(X_i) = \mathcal{R}\left(\sum_{j \in N_i} h^{T}_{j \to i}\right)
    \end{equation}
\end{minipage}
\hfill
\begin{minipage}[t]{.3\textwidth}
    \textsc{Variable Node Mapping} \\
    \\
    \begin{equation}
    m^{(t+1)}_{i \to j} = \mathcal{M}\left({h}^t_i, {h}^t_j, e_{ij}\right)
    \end{equation}
    \begin{equation}
    m^{(t+1)}_{i} = \sum_{j \in N_{i}} m^{(t+1)}_{j \to i}
    \end{equation}
   \\
    \begin{equation}
    h_{i \to j}^{(t+1)} = \mathcal{U}(h_{i}^{(t)}, \bm{m}_{i}^{(t + 1)})
    \end{equation}
   \\
    \begin{equation}
    \hat{p}_i(X_i) = \mathcal{R}(h^{T}_i)
    \end{equation}
\end{minipage}
}


%----------------------------------------------------------------------------------------
%	EXPERIMENTS
%----------------------------------------------------------------------------------------
% \headerbox{Experiments}{name=experiments,span=2,column=1,row=1, below=mapping}{
% \begin{minipage}{0.3\textwidth}
%     \subsection*{Graphs}
%     \begin{center}
%         \centering
%         \includegraphics[width=0.2\linewidth]{./graph_imgs/Star.png}
%         \includegraphics[width=0.2\linewidth]{./graph_imgs/Path.png}
%     \end{subfigure}
    
%     Tree Graphs [1,2]
%     \end{center}
    
%     \begin{center}
%     \begin{subfigure}
%         \includegraphics[width=0.2\linewidth]{./graph_imgs/Cycle.png}
%         \includegraphics[width=0.2\linewidth]{./graph_imgs/Grid.png}
%         \includegraphics[width=0.2\linewidth]{./graph_imgs/Wheel.png}
%         \includegraphics[width=0.2\linewidth]{./graph_imgs/FC.png}
%     \end{subfigure}

%     Cycle Graphs [3, 4, 5, 6]
%     \end{center}
%     % \includegraphics[width = .9\linewidth]{figures/graphs.png}
% \end{minipage} \hfill
% \begin{minipage}{.6\textwidth}
%     \subsection*{Models}
%     \begin{itemize}
%     \item Baseline:
%     \begin{itemize}
%         \item Belief Propagation
%     \end{itemize}
%     \item For each mapping method:
%     \begin{itemize}
%         \item GNN with GRU/LSTM Update Function
%     \end{itemize}
%     \item MGNN with GRU and Soft Attention 
%     \end{itemize}
% \end{minipage}
% }


%----------------------------------------------------------------------------------------
%	RESULTS
%----------------------------------------------------------------------------------------
\headerbox{Experiments \& Results}{name=results,span=2,column=1,below=mapping}{ 
\begin{minipage}[t]{0.5\textwidth}
    \textbf{Graphs}
    \begin{center}
    \begin{subfigure}
        \centering
        \includegraphics[width=0.25\linewidth]{./graph_imgs/Star.png}
        \includegraphics[width=0.25\linewidth]{./graph_imgs/Path.png}
        \includegraphics[width=0.25\linewidth]{./graph_imgs/Cycle.png}
        \includegraphics[width=0.25\linewidth]{./graph_imgs/Grid.png}
        \includegraphics[width=0.25\linewidth]{./graph_imgs/Wheel.png}
        \includegraphics[width=0.25\linewidth]{./graph_imgs/FC.png}
    \end{subfigure}
    
    Graph structure types: star, path, cycle, grid, wheel, fc
    \end{center}
\end{minipage} \hfill
\begin{minipage}[t]{.5\textwidth}
    \textbf{In Sample Performance}
    \begin{center}
        \includegraphics[width=.9\linewidth]{./poster_imgs/insample_perf.png}
    \end{center}
\end{minipage} \hfill
\begin{minipage}{.5\textwidth}
    \textbf{Combined Dataset Performance}
    \begin{center}
        \includegraphics[width=.9\linewidth]{./poster_imgs/combined_perf.png}
    \end{center}  
\end{minipage} \hfill
\begin{minipage}{.5\textwidth}
    \textbf{Inference Time}
    \begin{center}
        \includegraphics[width=.9\linewidth]{./poster_imgs/inference_time.png}
    \end{center} 
\end{minipage}
}

% \subsection*{Scatter plots}
%     % \includegraphics[width = .9\linewidth]{figures/graphs.png}
%     \begin{center}
%     \begin{subfigure}
%         \centering
%         \includegraphics[width=0.1\linewidth]{plots/star/res_mgnn_inference_star_small_star_small_bp.png}
%         \includegraphics[width=0.1\linewidth]{plots/path/res_mgnn_inference_path_small_path_small_bp.png}
%         \includegraphics[width=0.1\linewidth]{plots/grid/res_grid_small_mgnn_inference_bp.png}

%         \includegraphics[width=0.1\linewidth]{plots/star/res_mgnn_inference_star_small_star_small_gnn.png}   
%         \includegraphics[width=0.1\linewidth]{plots/path/res_mgnn_inference_path_small_path_small_gnn.png}    \includegraphics[width=0.1\linewidth]{plots/grid/res_grid_small_mgnn_inference_gnn.png}
        
%         \includegraphics[width=0.1\linewidth]{plots/star/res_factor_gnn_inference_star_small_star_small_gnn.png}
%         \includegraphics[width=0.1\linewidth]{plots/path/res_factor_gnn_inference_path_small_path_small_gnn.png}
%         \includegraphics[width=0.1\linewidth]{plots/grid/res_grid_small_factor_gnn_inference_gnn.png}
%     \end{subfigure}

%     True vs Expected Marginal for Small Graphs with BP, MGNN and FGNN (top to bottom) 
%     \end{center}


%----------------------------------------------------------------------------------------
%	CONCLUSION
%----------------------------------------------------------------------------------------
% \headerbox{Conclusion}{name=conclusion,column=1,below=results,span=2}{

% }


%----------------------------------------------------------------------------------------
%	REFERENCES
%----------------------------------------------------------------------------------------

%\headerbox{References}{name=references,column=2,below=application}{

%\smaller % Reduce the font size in this block
%\renewcommand{\section}[2]{\vskip 0.05em} % Get rid of the default "References" section title
%\nocite{*} % Insert publications even if they are not cited in the poster

%\bibliographystyle{unsrt}
%\bibliographystyle{IEEEtran}
%\bibliography{biblio} % Use biblio.bib as the bibliography file
%}
\headerbox{Conclusion \& Future Developments}
{name=conclusion,column=1,below=results, above=bottom,span=2}{
\begin{itemize}[noitemsep]
    \item GNN models are more robust to increasing graph complexity w.r.t. computation time
    \item GNNs are great candidates when increasing the complexity of graph, with no significantly or consistantly optimal model
    
\textbf{Potential Developments:} applying attention during $\mathcal{U}$; to other model structures (not Ising)
\end{itemize}
\vspace{.25pt}
} 

%----------------------------------------------------------------------------------------
%	ACKNOWLEDGEMENTS
%----------------------------------------------------------------------------------------

\headerbox{Acknowledgements}{name=acknowledgements,column=0,below=background, above=bottom,span=1}{
\scriptsize
We acknowledge the work of Yoon et al., "Inference in Probabilistic Graphical Models by Graph Neural Networks", 2019, and the use of open-source code from https://github.com/sunfanyunn/FE-GNN/tree/master. 
} 

% \headerbox{Conclusion}{name=conclusion,column=1,below=results,span=1}{

% }

\end{poster}

\end{document}